\chapter{Technical Background}
\label{sec:state}

% Hier werden zwei wesentliche Aufgaben erledigt:

% 1. Der Leser muß alles beigebracht bekommen, was er zum Verständnis
% der späteren Kapitel braucht. Insbesondere sind in unserem Fach die
% Systemvoraussetzungen zu klären, die man später benutzt. Zulässig ist
% auch, daß man hier auf Tutorials oder Ähnliches verweist, die hier auf
% dem Netz zugänglich sind.

% 2. Es muß klar werden, was anderswo zu diesem Problem gearbeitet
% wird. Insbesondere sollen natürlich die Lücken der anderen klar
% werden. Warum ist die eigene Arbeit, der eigene Ansatz wichtig, um
% hier den Stand der Technik weiterzubringen? Dieses Kapitel wird von
% vielen Lesern übergangen (nicht aber vom Gutachter ;-), auch später
% bei Veröffentlichungen ist "Related Work" eine wichtige Sache.

% Viele Leser stellen dann später fest, daß sie einige der Grundlagen
% doch brauchen und blättern zurück. Deshalb ist es gut,
% Rückwärtsverweise in späteren Kapiteln zu haben, und zwar so, daß man
% die Abschnitte, auf die verwiesen wird, auch für sich lesen
% kann. Diese Kapitel kann relativ lang werden, je größer der Kontext
% der Arbeit, desto länger. Es lohnt sich auch! Den Text kann man unter
% Umständen wiederverwenden, indem man ihn als "Tutorial" zu einem
% Gebiet auch dem Netz zugänglich macht.

% Dadurch gewinnt man manchmal wertvolle Hinweise von Kollegen. Dieses
% Kapitel wird in der Regel zuerst geschrieben und ist das Einfachste
% (oder das Schwerste weil erste).

\ldots state of the art \ldots

\todo{write state}

\cleardoublepage

%%% Local Variables:
%%% TeX-master: "diplom"
%%% End:
