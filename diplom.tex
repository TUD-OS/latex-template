\documentclass[a4paper,abstract=true,twoside,listof=totoc,%
numbers=noenddot,bibliography=totoc,BCOR1.5cm,headsepline,DIV12,%
appendixprefix,final]{scrreprt}

% You should select either american or british instead of english here:
\usepackage[ngerman,english]{babel}
%\usepackage{lmodern}            % use lmodern fonts

% This is the font I use for figures. Should I use it for everything?
\usepackage[bitstream-charter]{mathdesign}
\usepackage{berasans}

% Font 2
%% Use scalable, PostScript Type 1 versions of the Computer Modern fonts.
%\usepackage{type1cm}

%% Replace the standard Computer Modern Typewriter font LaTeX uses
%% for monospace text with the PostScript font Adobe Courier.
%\usepackage{courier}
% End-of-Font 2

% BibLaTeX instead of BibTeX
\usepackage[backend=biber,style=alphabetic]{biblatex}
\addbibresource{own.bib}

\usepackage{textcomp}           % for \textmu (non-italic $\mu$)
\setcounter{secnumdepth}{3}     % limit enumeration depth
\setcounter{tocdepth}{1}        % limit TOC depth
%\usepackage[it,bf,tight,hang,raggedright]{subfigure}
\usepackage{varioref}           % nice refs
\usepackage{graphicx}
\usepackage{csquotes}
\usepackage{wrapfig}
\usepackage{setspace}           % line spacing
\makeatletter
\usepackage{array}
\usepackage{fancybox}           % provide nice boxes
\usepackage{units}              % unified way of setting values with units
\usepackage[clearempty]{titlesec}
\usepackage{listings}           % nice source code listings
\lstset{frame=shadowbox, rulesepcolor=\color{blue}}

\pagenumbering{Roman}

\usepackage{fancyvrb}           % algorithm-boxes
\usepackage{color}
\usepackage{pdfpages}

% remove "pagebackref" for the final version
\usepackage[pdftex,
citebordercolor={0.75 0.75 1},
filebordercolor={0.75 0.75 1},
linkbordercolor={0.75 0.75 1},
% pagebordercolor={0.75 0.75 1},
urlbordercolor={0.75 0.75 1},
pdfborder={0.75 0.75 1},plainpages=false,pdfpagelabels=true]{hyperref}
\hypersetup{%
  pdftitle={Dein Titel},
  pdfauthor={Otto Mustermann},
  pdfkeywords={foo, bar},
}

\usepackage{hyphenat}
% custom hyphenation
%\hyphenation{cSCAN SCAN SSTF SATF FIFO in-te-res-siert}
%\lefthyphenmin=3
%\righthyphenmax=3

% instead of sloppy
%\tolerance 1414
%\hbadness 1414
\tolerance 2414
\hbadness 2414
\emergencystretch 1.5em
\hfuzz 0.3pt
\widowpenalty=10000     % Hurenkinder
\clubpenalty=10000      % Schusterjungen
\vfuzz \hfuzz
\raggedbottom

% use nice footnote indentation
\deffootnote[1em]{1em}{1em}{\textsuperscript{\thefootnotemark}\,}

%23.06.98
%
%Generelles zu den aux-Files und contents:
%ein addcontentsline-Befehl schreibt eine Zeile in das aux-File,
%welches am Ende des Latex-Laufes abgearbeitet wird und dann die
%einzelnen Files, welche als 1.Argument fuer addcontentsline gegeben
%wurden, erzeugt. Allerdings wird es nur ausgef"uhrt, wenn der
%starttoc-Befehl gegeben wurde (aber am Ende des Laufes).
%Der starttoc-Befehl wertet ausserdem das entsprechende File (toc, lof,..)
%aus und benutzt dazu ein definiertes Makro, welches die Form
%'l@<zweites Argument des addcontentsline-Befehls>' hat.

%\newcounter{todocounter}

%Hier erfolgt die Definition des Standard-Todo-Listen-names.
%\newcommand{\listoftodoname}{Todo}

\newcommand{\Todo}[1]{%
%   \def\@tempc{\arabic{todocounter}}%
%   \textbf{$\clubsuit$} \marginpar{\footnotesize \emph{#1}}%
%   \addcontentsline{lod}
%                   {todo}
%                   {\protect \numberline{\ref{todo_label_\@tempc}}#1}%
%   \label{todo_label_\@tempc}%
%   \stepcounter{todocounter}%
}

\newcommand{\TodoLong}[2]{%
%   \def\@tempc{\arabic{todocounter}}%
%   \emph{#2} \marginpar{\footnotesize \emph{#1}}%
%   \addcontentsline{lod}
%                   {todo}
%                   {\protect \numberline{\ref{todo_label_\@tempc}}#1}%
%   \label{todo_label_\@tempc}%
%   \stepcounter{todocounter}%
}

\newcommand{\MissingSource}[1]{%
%  \TodoLong{Missing source: #1}{\textbf{$\clubsuit$}}%
}

%zunaechst wird getestet, ob chapters verfuegbar sind, dann wird ein solches
%eingef"uhrt, sonst eine section
%
%mittels @namedef wird fuer starttoc eine Bezeichner definiert, der fuer
%das Auslesen des Aux-files verwendet wird. Wird der Bezeichner hinter dem
% '@' angetroffen (bei \addcontentsline das zweite Argument), so wird das
%dahinter definierte genutzt. das zweite Argument bei \@dottedtocline
%scheint den Abstand vor, das dritte den nach einer Kapitelzahl/Numerierung
%anzugeben. Da die Todos nicht numeriert werden, ist es 0.
\newcommand{\listoftodos}{%
  \@ifundefined{chapter}{\def\@tempa{\section*}}%
                        {\def\@tempa{\chapter*}}%
  \@tempa{\listoftodoname}%
  \@namedef{l@todo}{\@dottedtocline{1}{0em}{3.5em}}%
  \@starttoc{lod}%
}

\usepackage{xspace}
% some common commands
\newcommand{\drops}{\texorpdfstring{\textsc{Drops}\xspace}{DROPS}}
\newcommand{\LLinux}{\texorpdfstring{L$\!^4$Linux}{L4Linux}}

\newcommand{\NOVA}{NOVA\xspace}
\newcommand{\QEMU}{QEMU\xspace}

% If you know when you will hand in your thesis, enter the date here.
%\date{30. April 2009}

\makeatother
\begin{document}
%\listoftodos{}

\begin{singlespace}

\subject{{\LARGE Diplomarbeit}}

\title{Dein Titel}

\author{Otto Mustermann}

\publishers{Technische Universität Dresden\\
Fakultät Informatik\\
Institut für Systemarchitektur\\
Professur Betriebssysteme\\
\begin{minipage}{\textwidth}%\\
\vskip 6cm
 {\normalsize }\begin{tabular}{ll}
Betreuender Hochschullehrer: &
Prof. Dr. rer. nat. Hermann Härtig\tabularnewline
Betreuender Mitarbeiter: &
Dipl.-Inf. Dein Betreuer\tabularnewline
\end{tabular} {\normalsize }\end{minipage}}

\maketitle
\end{singlespace}
\newpage

\includepdf{images/diplom-aufgabe.pdf}
\cleardoublepage

\selectlanguage{ngerman}
\section*{\vfill{} \thispagestyle{empty}
Erklärung}

Hiermit erkläre ich, dass ich diese Arbeit selbstständig erstellt
und keine anderen als die angegebenen Hilfsmittel benutzt habe.
\bigskip{}

\noindent Dresden, den \today % 30. April 2009
\vspace{2.5cm}

\noindent Otto Mustermann \cleardoublepage{}

% NOTE: if you selected british or american above, change that here too
\selectlanguage{english}

\begin{abstract}
\input{abstract.tex}
\end{abstract}

\cleardoublepage

\tableofcontents{}

\listoftables
\listoffigures

\newpage
% \begin{center}
%   This page is intentionally left blank.
% \end{center}
% Force LaTeX to really emit an empty page.
\vspace*{1em}
\cleardoublepage

%\doublespacing
\input{introduction.tex}
\input{state.tex}
\input{design.tex}
\input{implementation.tex}
\chapter{Evaluation}
\label{sec:evaluation}

% Zu jeder Arbeit in unserem Bereich gehört eine Leistungsbewertung. Aus
% diesem Kapitel sollte hervorgehen, welche Methoden angewandt worden,
% die Leistungsfähigkeit zu bewerten und welche Ergabnisse dabei erzielt
% wurden. Wichtig ist es, dem Leser nicht nur ein paar Zahlen
% hinzustellen, sondern auch eine Diskussion der Ergebnisse
% vorzunehmen. Sehr gut ist, wenn man zunächst diskutiert und plausibel
% macht, welche Ergebnisse man erwartet, und dann eventuelle
% Abweichungen diskutiert.

\ldots evaluation \ldots

\cleardoublepage

%%% Local Variables:
%%% TeX-master: "diplom"
%%% End:

\chapter{Future Work}
\label{sec:futurework}

\ldots future work \ldots

\cleardoublepage

%%% Local Variables:
%%% TeX-master: "diplom"
%%% End:

\input{conclusion.tex}

\appendix

%\addchap{Glossar}

% makeglossaries diplom
%\printglossary[style=altlist]
%\printglossary[type=\acronymtype,style=long]

\printbibliography

\end{document}
