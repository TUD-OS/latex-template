\documentclass[a4paper,abstract=true,twoside,listof=totoc,%
numbers=noenddot,bibliography=totoc,BCOR1.5cm,headsepline,DIV12,%
appendixprefix,final]{scrreprt}

% You should select either american or british instead of english here:
\usepackage[ngerman,english]{babel}
\usepackage{fontspec}

% Font 2
%% Use scalable, PostScript Type 1 versions of the Computer Modern fonts.
%\usepackage{type1cm}

%% Replace the standard Computer Modern Typewriter font LaTeX uses
%% for monospace text with the PostScript font Adobe Courier.
%\usepackage{courier}
% End-of-Font 2

% BibLaTeX instead of BibTeX
\usepackage[backend=biber,style=alphabetic,alldates=long]{biblatex}
\addbibresource{own.bib}

\setcounter{secnumdepth}{3}     % limit enumeration depth
\setcounter{tocdepth}{1}        % limit TOC depth
 % create subfigures (e.g. Fig. 1a, 1b, 1c ...) in one figure environment:
\usepackage[it,bf,tight,hang,raggedright]{subfigure}
\usepackage{varioref}           % nice refs
\usepackage{graphicx}
\usepackage{csquotes}
\usepackage{wrapfig}
\usepackage{setspace}           % line spacing
\makeatletter
\usepackage{array}
\usepackage{fancybox}           % provide nice boxes
\usepackage{units}              % unified way of setting values with units
% apparently breaks KOMA-Script:
%\usepackage[clearempty]{titlesec}
\usepackage{listings}           % nice source code listings
\usepackage{color}
\usepackage{booktabs}           % nice tables
\usepackage{microtype}          % better looking text borders
\usepackage{ellipsis}		% better spacing for ellipsis (...)

\definecolor{mygreen}{rgb}{0,0.6,0}
\definecolor{mygray}{rgb}{0.5,0.5,0.5}
\definecolor{mymauve}{rgb}{0.58,0,0.82}

\lstset{ %
  frame=shadowbox,
  rulesepcolor=\color{blue},
  backgroundcolor=\color{white},   % choose the background color; you must add \usepackage{color} or \usepackage{xcolor}
%  basicstyle=\footnotesize,        % the size of the fonts that are used for the code
  breakatwhitespace=false,         % sets if automatic breaks should only happen at whitespace
  breaklines=true,                 % sets automatic line breaking
  captionpos=b,                    % sets the caption-position to bottom
  commentstyle=\color{mygreen},    % comment style
  deletekeywords={...},            % if you want to delete keywords from the given language
  escapeinside={\%*}{*)},          % if you want to add LaTeX within your code
  extendedchars=true,              % lets you use non-ASCII characters; for 8-bits encodings only, does not work with UTF-8
  frame=single,                    % adds a frame around the code
  keepspaces=true,                 % keeps spaces in text, useful for keeping indentation of code (possibly needs columns=flexible)
  keywordstyle=\color{blue},       % keyword style
  language=C,                 % the language of the code
 % morekeywords={*,...},            % if you want to add more keywords to the set
  numbers=left,                    % where to put the line-numbers; possible values are (none, left, right)
  numbersep=7pt,                   % how far the line-numbers are from the code
  numberstyle=\tiny\color{mygray}, % the style that is used for the line-numbers
  rulecolor=\color{black},         % if not set, the frame-color may be changed on line-breaks within not-black text (e.g. comments (green here))
  showspaces=false,                % show spaces everywhere adding particular underscores; it overrides 'showstringspaces'
  showstringspaces=false,          % underline spaces within strings only
  showtabs=false,                  % show tabs within strings adding particular underscores
  stepnumber=1,                    % the step between two line-numbers. If it's 1, each line will be numbered
  stringstyle=\color{mymauve},     % string literal style
  tabsize=2,                       % sets default tabsize to 2 spaces
  title=\lstname                   % show the filename of files included with \lstinputlisting; also try caption instead of title
}

\pagenumbering{Roman}

\usepackage{fancyvrb}           % algorithm-boxes
\usepackage{color}
\usepackage{pdfpages}

% remove "pagebackref" for the final version
\usepackage[pdftex,
citebordercolor={0.75 0.75 1},
filebordercolor={0.75 0.75 1},
linkbordercolor={0.75 0.75 1},
% pagebordercolor={0.75 0.75 1},
urlbordercolor={0.75 0.75 1},
pdfborder={0.75 0.75 1},plainpages=false,pdfpagelabels=true]{hyperref}
\hypersetup{%
  pdftitle={Dein Titel},
  pdfauthor={Otto Mustermann},
  pdfkeywords={foo, bar},
}

\usepackage{hyphenat}
% custom hyphenation
%\hyphenation{cSCAN SCAN SSTF SATF FIFO in-te-res-siert}
%\lefthyphenmin=3
%\righthyphenmax=3

% instead of sloppy
%\tolerance 1414
%\hbadness 1414
\tolerance 2414
\hbadness 2414
\emergencystretch 1.5em
\hfuzz 0.3pt
\widowpenalty=10000     % Hurenkinder
\clubpenalty=10000      % Schusterjungen
\vfuzz \hfuzz
\raggedbottom

% use nice footnote indentation
\deffootnote[1em]{1em}{1em}{\textsuperscript{\thefootnotemark}\,}

\usepackage{todonotes}

\usepackage{xspace}

% use this one last
% (redefines some macros for compatibility with KOMAScript)
\usepackage{scrhack}

% some common commands
\newcommand{\drops}{\texorpdfstring{\textsc{Drops}\xspace}{DROPS}}
\newcommand{\LLinux}{\texorpdfstring{L$\!^4$Linux}{L4Linux}}

\newcommand{\NOVA}{NOVA\xspace}
\newcommand{\QEMU}{QEMU\xspace}

% If you know when you will hand in your thesis, enter the date here.
%\date{30. April 2009}
%\newcommand{\printdate}{\@date}

\makeatother
\begin{document}
% use this to generate a list of all todo markers
% \listoftodos{}

% The intro of the work has to be written in German.
\selectlanguage{ngerman}

\begin{singlespace}

\subject{{\LARGE Diplomarbeit}}

\title{Dein Titel}

\author{Otto Mustermann}

\publishers{Technische Universität Dresden\\
Fakultät Informatik\\
Institut für Systemarchitektur\\
Professur Betriebssysteme\\
\begin{minipage}{\textwidth}%\\
\vskip 6cm
 {\normalsize }\begin{tabular}{ll}
Betreuender Hochschullehrer: &
Prof. Dr. rer. nat. Hermann Härtig\tabularnewline
Betreuender Mitarbeiter: &
Dipl.-Inf. Dein Betreuer\tabularnewline
\end{tabular} {\normalsize }\end{minipage}}

\maketitle
\end{singlespace}
\newpage

\includepdf{images/diplom-aufgabe.pdf}
\cleardoublepage

\section*{\vfill{} \thispagestyle{empty}
Erklärung}

Hiermit erkläre ich, dass ich diese Arbeit selbstständig erstellt
und keine anderen als die angegebenen Hilfsmittel benutzt habe.
\bigskip{}

\noindent Dresden, den \today % \printdate % if you defined date earlier
\vspace{2.5cm}

\noindent Otto Mustermann, äöüßÄÖÜ \cleardoublepage{}

% NOTE: if you selected british or american above, change that here too
\selectlanguage{english}

\begin{abstract}
\input{abstract.tex}
\end{abstract}

\cleardoublepage

\tableofcontents{}

\listoftables
\listoffigures

\newpage
% \begin{center}
%   This page is intentionally left blank.
% \end{center}
% Force LaTeX to really emit an empty page.
\vspace*{1em}
\cleardoublepage

%\doublespacing
\input{introduction.tex}
\input{state.tex}
\input{design.tex}
\input{implementation.tex}
\chapter{Evaluation}
\label{sec:evaluation}

% Zu jeder Arbeit in unserem Bereich gehört eine Leistungsbewertung. Aus
% diesem Kapitel sollte hervorgehen, welche Methoden angewandt worden,
% die Leistungsfähigkeit zu bewerten und welche Ergabnisse dabei erzielt
% wurden. Wichtig ist es, dem Leser nicht nur ein paar Zahlen
% hinzustellen, sondern auch eine Diskussion der Ergebnisse
% vorzunehmen. Sehr gut ist, wenn man zunächst diskutiert und plausibel
% macht, welche Ergebnisse man erwartet, und dann eventuelle
% Abweichungen diskutiert.

\ldots evaluation \ldots

\cleardoublepage

%%% Local Variables:
%%% TeX-master: "diplom"
%%% End:

\chapter{Future Work}
\label{sec:futurework}

\ldots future work \ldots

\cleardoublepage

%%% Local Variables:
%%% TeX-master: "diplom"
%%% End:

\input{conclusion.tex}

\appendix

%\addchap{Glossar}

% makeglossaries diplom
%\printglossary[style=altlist]
%\printglossary[type=\acronymtype,style=long]

\printbibliography
\iffalse
    % an aid for Kile autocompletion
    \bibliography{own.bib}
\fi

\end{document}
